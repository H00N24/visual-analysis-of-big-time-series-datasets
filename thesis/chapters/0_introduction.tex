\chapter*{Introduction}
\addcontentsline{toc}{chapter}{Introduction}
Nowadays, there are only a few aspects of everyday life that are not affected by modern technologies, and the rise of technologies like \textit{Internet of Things} lowers this number even more. These technologies produce large amounts of data with temporal stamp, also recognized as time series. As the amount and speed, at which we are getting new data, are making it impossible to process and analyze them manually, we are in need for fast analytics tools for time series data.

Our work focuses on methods for analyzing large datasets of long time series, demonstrating it on a power consumption dataset.  We are using the approach to transform the dataset into a spatial feature space representation. Using this representation, we prepare multiple visualizations for exploring the structures within the dataset and detailed views for examining time series details. To supply the analysis, we apply clustering and anomaly detection methods. This work is divided into four chapters.

Chapter 1 will discuss metrics used for time series, their complexity, advantages and disadvantages, and their application in practical analysis. Afterwards, it explores the representations of time series datasets in artificial feature space. Finally, it examines clustering and anomaly detection methods available for large datatasets.

Chapter 2 will discuss techniques for visual representation and analysis of time series datasets. Firstly, it analyzes the dimensionality reduction techniques for transforming the datasets into low-dimensional embeddings. Secondly, it goes through algorithms for downsampling time series with a specific focus on visual analysis.

Chapter 3 applies the methods from the first two chapters to analyze a large power consumption dataset, while introducing a novel approach for time series transformation.

Chapter 4 is a conclusion of our work, containing the discussion of the benefits and drawbacks of our novel approach about the novel approach for time series transformation, and possible future research directions in this field.